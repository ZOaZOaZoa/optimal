\section{ Задача 1 }

\subsection{Условия задачи}

Задано дифференциальное уравнение описывающее объект, при известных граничных условиях $\b{x}(t_0) = \b{x}_0,\,\b{x}(t_1) = \b{x}_1$:
\begin{equation}
	\dddot{x} + 4\ddot{x} + 4\dot{x} + x = u_1
\end{equation}

На управление наложено ограничение $|u_1| \leqslant 1$. А также введены изопериметрические ограничения:
\begin{eqnarray}
	\int_{t_0}^{t_1} \left( 2x_1 + x_2 \right) dt = 0 \\
	\int_{t_0}^{t_1} \left( 3x_1 + 4x_2 \right) dt = 0
\end{eqnarray}

При таких условиях нужно оптимизировать следующий критерий:
\begin{equation}
	J(t_0,\,t_1) = \int_{t_0}^{t_1} dt = t_1 - t_0 \to \min_{|u_1|\leqslant1}
\end{equation}

\subsection{Решение задачи}

Получим представление объекта в форме задачи Коши $\dot{\b{x}} = f(\b{x}, u_1)$:
\begin{equation}
	\begin{cases}
		\dot{x}_1 = x_2  \\
		\dot{x}_2 = x_3 \\
		\dot{x}_3 = -x_1 -4x_2 - 4x_3 + u_1
	\end{cases}
	\label{cosh1}
\end{equation}

Получим гамильтониан. Так как рассматривается задача быстродействия, в соответствии с видом $J$, в выражении гамильтониана отсутствует слагаемое $\psi_0 G_0$. Для задачи быстродействия полагается $\psi_0 = 0$.
\begin{multline}
	H = - \bm{\lambda}^\intercal \cdot \b{G} + \bm{\psi}^\intercal \cdot f(\b{x}, u_1) = \\ 
	= -\lambda_1 (2x_1 + x_2) - \lambda_2 (3x_1 + 4x_2) + \psi_1 x_2 + \psi_2 x_3 + \psi_3 (-x_1 - 4x_2 - 4x_3 + u_1)
\end{multline}

Получим значение Лагранжиана:
\begin{multline}
	\mathcal{L} = -H + \bm{\psi}^\intercal \cdot \dot{\b{x}} = \bm{\lambda}^\intercal \cdot \b{G} + \bm{\psi}^\intercal \cdot \left(\dot{\b{x}} - f(\b{x},\,u_1)\right) = \\ =
	\lambda_1 (2x_1 + x_2) + \lambda_2 (3x_1 + 4x_2) + \psi_1 (\dot{x}_1 - x_2) + \psi_2(\dot{x}_2 - x_3) + \psi_3 (\dot{x}_3 + x_1 + 4x_2 + 4x_3 - u_1)
\end{multline}

Получим уравнения сопряжённого состояния $\dot{\bm{\psi}} = -\pdv{H}{\b{x}}$:
\begin{equation}
	\begin{cases}
		\dot{\psi}_1 = 2\lambda_1 + 3\lambda_2 + \psi_3 \\
		\dot{\psi}_2 = \lambda_1 + 4\lambda_2 - \psi_1 + 4\psi_3 \\ 
		\dot{\psi}_3 = -\psi_2 + 4\psi_3
	\end{cases}
\end{equation}

Как известно, если управление входит в гамильтониан линейно, то есть
\begin{equation*}
	H  = \bm{\varphi}(\bm{\psi},\,\b{x}) + \bm{\psi}^\intercal B \b{u}
\end{equation*}
то $j$-ая компонента оптимального управления $u_j^{*}$ определяется видом ограничения $|u_j|\leqslant\beta_j$ и значением $\left(\bm{\psi}^\intercal B\right)_j$ -- $j$-м элементом данной вектор-строки:
\begin{equation}
	u_j^{*} = \beta_j\, \mathrm{sign}\,\left[ \left(\bm{\psi}^\intercal B \right)_j \right] \Rightarrow u_1^{*} = \mathrm{sign}\,\psi_3
\end{equation}

Подставим в (\ref{cosh1}) и получим уравнения, описывающие оптимальный процесс в объекте:
\begin{equation}
	\begin{cases}
		\dot{x}_1 = x_2  \\
		\dot{x}_2 = x_3 \\
		\dot{x}_3 = -x_1 -4x_2 - 4x_3 + \mathrm{sign}\,\psi_3
	\end{cases}
\end{equation}

\section{Задача 2}

\subsection{Условия задачи}

Задано дифференциальное уравнение, описывающее объект, при известной левой границе $\b{x}(t_0) = \b{x}_0$. При этом значения $\b{x}(t_1),\,t_1$ -- неизвестны:
\begin{equation}
	\dddot{x} + 2\ddot{x}+ \dot{x} + 5x = 2u_1 + 5u_2
\end{equation}

На управление наложены ограничения $\b{u}\in\Omega_u$: $|u_1|\leqslant 5$ и $|u_2| \leqslant 2$. А также введены изопериметрические ограничения:
\begin{eqnarray}
	\int_{t_0}^{t_1} (5x_1 + 5x_2)dt + 4t_1 + 3x_1(t_1) + 3x_2(t_1) + 3x_3(t_1) = 0 \\ 
	\int_{t_0}^{t_1} (3x_1 + x_2)dt + 4t_1 + 4x_1(t_1) + 5x_2(t_1) + 2x_3(t_1) = 0
\end{eqnarray}

При таких условиях нужно оптимизировать следующий критерий:
\begin{equation}
	J = \int_{t_0}^{t_1} (2x_1 + x_2 + 4x_3 + 3u_1^2 + 5u_2^2)dt + 2t_1 + 3x_1(t_1) + 4x_2(t_1) + 2x_3(t_1) \to \min_{\b{u}\in\Omega_u}
\end{equation}

\subsection{Решение задачи}

Получим представление объекта в форме задачи Коши $\dot{\b{x}} = f(\b{x}\,\b{u})$:
\begin{equation}
	\begin{cases}
		\dot{x}_1 = x_2 \\
		\dot{x}_2 = x_3 \\
		\dot{x}_3 = -5x_1 - x_2 - 2x_3 + 2u_1+ 5u_2
	\end{cases}
\end{equation}


Получим гамильтониан:
\begin{multline}
	H = -\lambda_0 G_0 - \bm{\lambda}^\intercal \cdot \b{G} + \bm{\psi}^\intercal \cdot f(\b{x}, u_1) = \\ 
	= -\lambda_0 (2x_1 + x_2 + 4x_3 + 3u_1^2 + 5u_2^2) - \lambda_1 (5x_1 + 5x_2) - \lambda_2(3x_1 + x_2) + \\ + \psi_1 x_2 + \psi_2 x_3 + \psi_3 (-5x_1 - x_2 - 2x_3 + 2u_1 + 5u_2)
\end{multline}

Получим значение лагранжиана:
\begin{multline}
	\mathcal{L} = -H + \bm{\psi}^\intercal \cdot \dot{\b{x}} =  \lambda_0 G_0 + \bm{\lambda}^\intercal \cdot \b{G} + \bm{\psi}^\intercal \cdot \left(\dot{\b{x}} - f(\b{x},\,u_1)\right) = \\
	= \lambda_0 (2x_1 + x_2 + 4x_3 + 3u_1^2 + 5u_2^2) \lambda_1 (5x_1 + 5x_2) \lambda_2(3x_1 + x_2) + \\
	+ \psi_1(\dot{x}_1 - x_2) + \psi_2(\dot{x}_2 - x_3) + \psi_3(\dot{x}_3 + 5x_1 + x_2 + 2x_3 - u_1 - 5u_2)
\end{multline}

Получим уравнения сопряжённого состояния $\dot{\bm{\psi}} = -\pdv{H}{\b{x}}$:
\begin{equation}
	\begin{cases}
		\dot{\psi}_1 = 2\lambda_0 + 5\lambda_1 + 3\lambda_2 + 5\psi_3 \\ 
		\dot{\psi}_2 = \lambda_0 + 5\lambda_1 + \lambda_2 - \psi_1 + \psi_3 \\
		\dot{\psi}_3 = 4\lambda_0 - \psi_2 + 2\psi_3
	\end{cases}
\end{equation}

Так как значение $\b{x}(t_1)$ -- неизвестно, введём условие трансверсальности $\bm{\psi}(t_1) = -\pdv{T}{\b{x}(t_1)}$:
\begin{equation}
	\begin{cases}
		\psi_1(t_1) = -3\lambda_0 -3\lambda_1 - 4\lambda_2  \\
		\psi_2(t_1) = -4\lambda_0 - 3\lambda_1 - 5\lambda_2 \\ 
		\psi_3(t_1) = -2\lambda_0 - 3\lambda_1 - 2\lambda_2
	\end{cases}
\end{equation}

Так как значение $t_1$ -- неизвестно, введём условие стационарности по $t_1$, имеющее следующий вид $\pdv{J_\mathcal{L}}{t_k} = (-1)^{k+1}\lambda_0 G_0(t_k) + \bm{\lambda}^\intercal \b{G}(t_k) + \dv{T}{t_k}(t_k) = 0$:
\begin{multline}
	\lambda_0 G_0(t_1) + \lambda_1 G_1(t_1) + \lambda_2 G_2(t_1) + \lambda_0 (2 + 3\dot{x}_1(t_1) + 4\dot{x}_2(t_1) + 2\dot{x}_3(t_1)) + \\
	+ \lambda_1 (4 + 3\dot{x}_1(t_1) + 3\dot{x}_2(t_1) + 3\dot{x}_3(t_1)) + \lambda_2 (4 + 4\dot{x}_1(t_1) + 5\dot{x}_2(t_1) + 2\dot{x}_3(t_1)) = 0
\end{multline}

Так как управление входит в гамильтониан нелинейно, найдём оптимальное управление как кусочную функцию из условия стационарности по $\b{u}$, $\pdv{H}{\b{u}} = 0$:
\begin{eqnarray}
	u_1^{*} = \begin{cases}
		\cfrac{2\psi_3}{6\lambda_0}, & |u_1| < 5 \\
		5\,\mathrm{sign}\,2\psi_3, & |u_1| \geqslant 5
	\end{cases} \\
	u_2^{*} = \begin{cases}
		\cfrac{5\psi_3}{10\lambda_0}, & |u_2| < 2 \\
		2\,\mathrm{sign}\,5\psi_3, & |u_2| \geq 2
	\end{cases}
\end{eqnarray}


