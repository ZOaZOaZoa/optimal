\section{ Задача (2b) }

\subsection{Условия задачи}

Требуется определить структуру регулятора, который обеспечит оптимальный режим движения объекта с передаточной функцией:
\begin{equation*}
	W(p) = \frac{X(p)}{U(p)} = \frac{4}{p}
\end{equation*}

Критерий оптимизации:
\begin{equation*}
	J = \int_0^\infty \left( 18 x^2 + 8u^2 \right) dt \to \min_{u}
\end{equation*}

Краевые условия:
\begin{equation*}
	x(\infty) = 0,\;x(0)=5
\end{equation*}

\subsection{Решение}

Нужно синтезировать регулятор для использования его в соответствии со структурной схемой, представленной на рис. \ref{reg_scheme}. Будем искать передаточную функцию регулятора $K(p)$.

\begin{center}
	\begin{tikzpicture}
		\node(Wp)[block] at (0,0) {$W(p)$};
		\node(reg)[block, below=1.3] at (Wp) {$K(p)$};
		
		\draw [arrow] (reg) -- +(-2, 0) node [midway, above] {$U(p)$} |- (Wp);
		\draw [arrow] (Wp) -- +(2, 0) node (x)[circle, fill, scale=0.3] {} |- (reg);
		\draw [arrow] (x) -- +(2, 0) node [midway, above] {$X(p)$};
	\end{tikzpicture}
	\captionof{figure}{Структурная схема объекта с синтезируемым регулятором}
	\label{reg_scheme}	
\end{center}

Получим безусловное ограничение из вида передаточной функции объекта:
\begin{equation*}
	W(p) = \frac{4}{p} \Leftrightarrow pX(p) = 4U(p) \stackrel{\mathscr{L}}{\to} \dot{x}(t) = 4u(t)
\end{equation*}
\begin{equation}
	\begin{cases}
		\dot{x}(t) = 4u(t) \\
		x(0) = 5,\; x(\infty) = 0 
	\end{cases}
\end{equation}

Запишем гамильтониан для данной системы, чтобы свести задачу к безусловной оптимизации:
\begin{equation}
	H(x,u,\psi) = -\lambda_0 \left( 18x^2 + 8u^2 \right) + 4\psi u 
\end{equation}

Получим оптимальное уравнение, для этого запишем связь с лагранжианом, и воспользуемся условием стационарности по $u$:
\begin{equation}
	H = -\mathcal{L} + \psi \dot{x} \Rightarrow \pdv{H}{u} = -\pdv{\mathcal{L}}{u} + \psi \cancelto{0}{\pdv{\dot{x}(t)}{u}}
\end{equation}
Подставим значения и учтём стационарность по $u: \pdv{\mathcal{L}}{u} = 0$
\begin{equation}
	-16\lambda_0 u + 4\psi = 0 \Rightarrow u = \frac{\psi}{4\lambda_0} \stackrel{\lambda_0 = 1}{\Rightarrow} \frac{\psi}{4}
\end{equation}

Для нахождения оптимального управления $u^*(x)$ запишем уравнения Эйлера-Лагранжа в канонической форме:
\begin{equation*}
	\begin{cases}
		\dot{\psi} = -\pdv{H}{x} = 36x \\
		\dot{x} = \pdv{H}{\psi} = 4u = \psi
	\end{cases}
\end{equation*} 

Характеристическое уравнение можно получить из информации о матрице состояния данной системы:
\begin{equation}
	\begin{vmatrix}
		-p  & 36 \\
		1 & -p
	\end{vmatrix} = p^2-36 = 0 \Leftrightarrow p_{1,\,2} = \pm 6
\end{equation}

Тогда переменная состояния может быть представлена в виде:
\begin{equation}
	x(t) = C_1 e^{6t} + C_2 e^{-6t}
\end{equation}

Из условия $x(\infty) = 0$ получаем $C_1 = 0$. А из другого граничного условия $x(0) = 5$ получаем $C_2 = 5$.

Теперь получим зависимость $u^{*}(x)$. Для этого воспользуемся безусловным ограничением:
\begin{equation}
	\dot{x} = -6\cdot C_2 e^{-6t} = -6x = 4u \Rightarrow u^{*}(x) = -\frac{3}{2}x
\end{equation}

Итого получаем, что оптимальным регулятором будет звено усиления с передаточной функцией $K(p) = -\frac{3}{2}$.



\section{Задача (6c)}
\subsection{Условия задачи}

Найти оптимальное управление и оптимальную траекторию в задаче:
\begin{equation*}
	\dot{x} = 2,5u
\end{equation*}
\begin{equation*}
	x(0) = 0,\;t\in[0,\;1,5]\;\text{(второй конец не закреплен)}
\end{equation*}
\begin{equation*}
	|u| \leqslant 4
\end{equation*}
\begin{equation*}
	\int_0^{1,5} \left( 2,5u^2 + 10x \right) dt \to \min_{u\in\Omega_u}
\end{equation*}

\subsection{Решение}

Составим гамильтониан:
\begin{equation}
	H(x,u) = -\lambda_0\left( 2,5u^2 + 10x \right) + 2,5\psi u
\end{equation}

Оптимальное управление выразим из условия стационарности по $u: \pdv{H}{u} = 0$ т.к. управление входит в гамильтониан нестационарно
\begin{equation}
	-5\lambda_0 u + 2,5\psi(t) = 0 \Rightarrow u = \frac{\psi(t)}{2\lambda_0} \stackrel{\lambda_0 = \frac{1}{2}}{=} \psi(t)
\end{equation}

В соответствии с принципом максимума оптимальное управление примет вид:
\begin{equation*}
	u^{*}(t)=
	\begin{cases}
		\psi(t), & |u|\leqslant 4 \\
		4\mathrm{sign}\, \psi(t), & |u| > 4
	\end{cases}
\end{equation*}

Запишем уравнение сопряжённого состояния $\dot{\psi} = -\pdv{H}{x}$:
\begin{equation}
	\dot{\psi} = 10\lambda_0 = 5 \Rightarrow \psi(t) = 5t + C
\end{equation}

В данной задаче терминант отсутствует $T=0$, поэтому условие трансверсальности примет вид $\psi(1,5) = -\pdv{T}{x(1,5)} = 0$. Отсюда следует, что $C = -7,5$.

Тогда управление примет вид:
\begin{equation}
	u^* (t) = 
	\begin{cases}
		-4, & t\in[0,\,0,7) \\
		5t - 7,5, & t\in[0,7,\; 1,5]
	\end{cases}
\end{equation}

Траектория будет иметь вид:
\begin{equation}
	x^* (t) =
	\begin{cases}
		10t + C_1, & t\in[0,\,0,7) \\
		\cfrac{12,5t^2}{2} - 2,5\cdot7,5t + C_2, & t\in[0,7,\; 1,5]
	\end{cases}
\end{equation}

Из условия $x(0) = 0$ получаем $C_1 = 0$. Значение $C_2$ найдем из непрерывности $x^* (t)$ в окрестности $t=0,7$.
\begin{equation}
	10\cdot 0,7 = \frac{12,5 \cdot 0,7^2}{2} - 2,5 \cdot 7,5 \cdot 0,7 + C_2 \Leftrightarrow C_2 = 17,0625
\end{equation}

Итого:
\begin{equation}
	x^* (t) =
	\begin{cases}
		10t, & t\in[0,\,0,7) \\
		\cfrac{12,5t^2}{2} - 2,5\cdot7,5t + 17,0625, & t\in[0,7,\; 1,5]
	\end{cases}
\end{equation}